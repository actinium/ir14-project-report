\section{Method}\label{method}

In this section the different components of the project are described. It gives an account of how the query completion module has been implemented. A frontend has also been implemented to provide a better environment for testing and demoing. Lastly, the section contains details on how the implementation was tested and what requirements the implementation should fulfill. 

\subsection{Implementation}

As previously mentioned, Apache Solr has built-in support for free text search. This functionality is provided by the class |org.apache.solr.spelling.suggest.Suggester| which is configured in the file |solrconfig.xml|. In order to implement our query suggestion, a new class, |SuggesterMK2|, has been implemented which piggybacks on the existing functionality of the original Suggester class.\cite{SUGGESTER}

The |SuggesterMK2| implementation attempts to be a drop-in replacement for the Suggester class. Notable exceptions to this general rule include:
\begin{itemize}

\item The |SuggesterMK2| uses the query parameter |spellcheck.q| instead of the parameter |q|
\item The |SuggesterMK2| uses a custom field in |solrconfig.xml| called |delegate| instead of the field called |field|.
\end{itemize}

The rationale for these seemingly arbitrary decisions will be stated explicitly in the coming sections of this report.

\subsubsection{Delegation}

The |SuggesterMK2| implementation uses the delegation pattern\cite{DELEGATE} in order to reuse the original suggester implementation as much as possible. Reusing the original Suggester class means that we can use its query suggestion implementation, as well as its loading and reloading implementations, all of which we can assume work as intended and would be difficult to implement from scratch.

\begin{figure}[h!]
    \centering
    \includegraphics[width=0.8\textwidth]{img/delegation.png}
    \caption{\footnotesize{Illustration of the delegation used by SuggesterMK2. The suggester will give suggestions drawn from the fields author and author\_fulltext for the field author, and from desc and desc\_case\_sensitive for the field desc.}}
    \label{fig:delegation}
\end{figure}

Let us say that we configure the system to give suggestions for the field |author| using the contents from the fields |author| and |author_exact|, the first of which being tokenized on whitespace, and the second being untokenized. All we have to do then, using delegation, is to construct a |Suggester| object for |author| and |author_exact|, query each of these, and then combine the results, merging results as needed if the results overlap. In our implementation, configuring the suggester to return results as in the diagram can be done by adding the following to the suggester configuration in |solrconfig.xml|:

\begin{verbatim}
<lst name="delegate">
  <str name="targetField">author</str>
  <str name="sourceField">author</str>
  <str name="sourceField">author_fulltext</str>
</lst>
\end{verbatim}

The fields we use for contents do not need to include the target field – we can use the contents from entirely different fields to use as suggestions. For a sample use case where this might be desired, consider what would happen if the target field was 4-gram tokenized: a word like ''parallelize'' would then be split up into the tokens ''para'', ''aral'', ''rall'', ''alle'', and so on, none of which we want to give as query suggestions. We could then draw suggestions from some other field where the contents are tokenized as words instead.

The |SuggesterMK2| implementation also gives query suggestions consisting of the field names it is configured to give suggestions for. For instance, the example configuration in figure \ref{fig:delegation} the |SuggesterMK2| would give |author:”| as a query suggestion when the user types |aut|.

For the full configuration of the suggester in |solrconfig.xml|, see Appendix.
For the implementation of the SuggesterMK2, see \cite{GITHUBPROJ}.

\subsubsection{Order of results}

The results of a query are returned in descending order of estimated relevance. Field names are given priority over field contents, while field contents and standard results are both sorted based on number of occurrences.

\subsubsection{Testing}

A simple frontend has been implemented to act as a tool for debugging and demoing the project. It uses JSON requests and responses to interact with the |/suggest| and |/select| features of the Solr server.

\begin{figure}[h!]
    \centering
    \includegraphics[width=0.8\textwidth]{img/frontend.png}
    \caption{\footnotesize{A query performed through the frontend, with query completions visible}}
    \label{fig:frontend}
\end{figure}

Testing was primarily done using the Solr Admin console and the frontend.
In the early stages of development, test data provided by Solr was used. This was later replaced with a larger set of documents containing fields for names, party affiliation and quotes from members of the Swedish parliament. A number of queries were used to test that the suggester worked as expected.
